\documentclass{minimal}
\usepackage{booktabs}
\usepackage{array}
\usepackage{lmodern}
\usepackage[letterpaper, margin=0.5in]{geometry}
\usepackage[utf8]{inputenc}
\renewcommand{\familydefault}{\sfdefault}
\begin{document}

%%%%%%%%%%%%%%%%
% Contact Info %
%%%%%%%%%%%%%%%%
\newcommand{\name}{First Last}
\newcommand{\address}{123 Evergreen Terrace\\Springfield, ND, 12345}
\newcommand{\phone}{1-234-567-8901}
\newcommand{\email}{somebody@someplace.com}
\newcommand{\github}{https://github.com/abcdef}

\begin{center}
\fontsize{14}{12.8}\selectfont
\name\\
\fontsize{10}{12}\selectfont
\address\\
\phone\\
\email\\
\github\\
\end{center}
\vspace{0.5cm}

%%%%%%%%%%
% Skills %
%%%%%%%%%%
\begin{tabular}{ p{1.5cm} p{1cm} p{16cm} }
\textbf{Habilidades:} & & \textbf{Lenguajes de Programación:} Python, C, C++, Matlab, Java, LabVIEW\\
& & \textbf{Microcontroladores:} MSP430, PIC16, SAMv71, Cyclone II (Nios II), OMAP, Snapdragon\\
& & \textbf{Sistemas Operativos Embedidos:} FreeRTOS, QNX, Yocto Linux, Angstrom Linux, Android\\
& & \textbf{Bibliotecas y Marcos de Trabajo:} OpenCV, PIL, D3, Matplotlib, Apex Chart, Scikit-learn, Numpy\\
& & \textbf{Otras Habilidades:} Visualización de Datos, Ágil, Planeamiento de Proyectos, Liderazgo Técnico\\
& & \\
\end{tabular}

%%%%%%%%%%%%%%
% Experience %
%%%%%%%%%%%%%%
\begin{tabular}{ p{1.5cm} p{1cm} p{16cm} >{\raggedleft\arraybackslash}p{3cm} }
\textbf{Experiencia:} & & \textbf{Mojo Vision, Saratoga, California} & \\
\end{tabular}

\begin{tabular}{ p{1.5cm} p{1cm} p{16cm} }
& & \textit{Ingeniero de Software Senior de Estado Mayor, Software, 2020}\\
& & • Establecer proceso de validación de software desde cero; asegurar comformidad a ISO 13485\\
& & • 'Dockerizar' construcciones de middleware para establecer construcciones de software repetibles\\
& & • Crear pruebas de aceptación SIL y HIL como parte de tubería de integración continua\\
& & • Escribir biblioteca en Python para controlar el equipo de HIL vía UART y SSH\\
& & \\
\end{tabular}

\begin{tabular}{ p{1.5cm} p{1cm} p{16cm} >{\raggedleft\arraybackslash}p{3cm} }
& & \textbf{Nio, San José, California} & \\
\end{tabular}

\begin{tabular}{ p{1.5cm} p{1cm} p{16cm} }
& & \textit{Ingeniero de Validación de Estado Mayor, Cabina Digital, 2018-2020}\\
& & • Líder de pruebas para la cabina digital de próxima generación; asegurar conformidad a ISO 26262\\
& & • Gestionar equipo de siete ingenieros por unos meses mientras supervisora estaba enferma\\
& & • Trabajar cercanamente con vendedores para calificar software y hardware de producto de caja\\
& & • Automatizar caracterización de velocidad de lectura-escritura UFS a través de SOs y hardware\\
& & • Desarrollar complemento de tablero para TestRail usando D3 para visualizar cobertura de pruebas\\
& & • Visualizar resultados de pruebas de estrés y longevidad usando Matplotlib; analizar los datos\\
& & • Entrenar nuevos empleados y equipos de China en uso de software y hardware de pruebas internos\\
& & • Crear biblioteca en Python y hardware de medición para detectar y comparar señales de audio\\
& & • Crear biblioteca en Python para remotamente controlar fuentes de alimentación del laboratorio\\
& & • Dirigir equipo de ingenieros para crear marco de trabajo de pruebas de palabras claves en Python\\
& & • Investigar y planificar estrategia para introducir pruebas dirigidas por modelos para HMI y MCU\\
& & \\
\end{tabular}

\begin{tabular}{ p{1.5cm} p{1cm} p{16cm} }
& & \textit{Ingeniero de Validación Senior, Cabina Digital, 2017-2018}\\
& & • Líder de pruebas para software de fabricación; asegurar calidad de software diagnóstico\\
& & • Escribir pruebas de memoria para DRAM y eMMC en C para indentificar y cuantificar errores soft\\
& & • Ayudar equipo de fabricación de China remotamente depurar problemas de pruebas usando LabVIEW\\
& & • Automatizar pruebas de longevidad para descubrir fallos de seguridad crítica de arranque\\
& & • Trabajar con desarolladores de software y vendedores para depurar fallos de arranque e I2C bus\\
& & • Establecer 50 bancos de pruebas para continuamente ejecutar pruebas de longevidad\\
& & • Utilizar OpenCV y Python para detectar video congelado y otros fallos del sistema de camera\\
& & • Crear biblioteca en Python para mandar y recibir mensajes de LIN utilizando el PCAN-LIN\\
& & • Crear biblioteca en Python para comunicar con tres SOs en dispositivo con uno puerta de UART\\
& & • Dirigir interna en proyecto de procesamiento de imágenes para clasificar imágenes en Python\\
& & • Representar oficina Americana en China para Salón del Automóvil de Pekín, y eventos en Shanghái\\
& & \\
\end{tabular}

\begin{tabular}{ p{1.5cm} p{1cm} p{10cm} >{\raggedleft\arraybackslash}p{3cm} }
& & \textbf{Apple, Cupertino, California} & \\
\end{tabular}

\begin{tabular}{ p{1.5cm} p{1cm} p{16cm} }
& & \textit{Ingeniero de QA de Software, Equipo de Aplicaciones, 2015-2017}\\
& & • Líder del equipo de aseguramiento de la calidad para 'Noticias en el Red'\\
& & • Probar nuevas funciones de ANF como tablas, paralaje, y artilugios de medios sociales\\
& & • Automatizar pruebas para la conversión de artículos del formato ANF para encontrar regresiones\\ 
& & • Escribir utilidad en Python para generar artículos para pruebas de estilo y tipo de letra\\
& & • Escribir utilidad usando Python y PIL para generar imágenes para probar la tubería de imágenes\\
& & \\
\end{tabular}

% Extra Space to Bump the following to the next page
\pagebreak

\begin{tabular}{ p{1.5cm} p{1cm} p{10cm} >{\raggedleft\arraybackslash}p{3cm} }
& & \textbf{Universidad McGill, Montréal, Quebec} & \\
\end{tabular}

\begin{tabular}{ p{1.5cm} p{1cm} p{16cm} }
& & \textit{Asistente de Enseñanza, Departamento de Ingenieria Eléctrica e Informática, 2013-2015}\\
& & • Desarrollar laboratorios de procesamiento de señales para la FPGA Altera DE2 y NI myRIO\\
& & • Enseñar alumnos como usar hardware FPGA, conceptos de diseño digital, y Verilog\\
& & • Implementar filtros digitales, codificadores convolucionales, y descifradores Viterbi\\
& & • Enseñar alumnos como usar LabVIEW; usarlo para implementar filtros digitales, FSMs, y clinómetro\\
& & • Dar clases sobre diseño de sistemas embedidos, redes de Petri, redes de Kahn, y planificadores\\
& & \\
\end{tabular}

\begin{tabular}{ p{1.5cm} p{1cm} p{10cm} >{\raggedleft\arraybackslash}p{3cm} }
& & \textbf{Texas Instruments, Dallas, Texas} & \\
\end{tabular}

\begin{tabular}{ p{1.5cm} p{1cm} p{16cm} }
& & \textit{Ingeniero de Producto Interno, Operaciones de Ingenieria Analógica, 2011-2013}\\
& & • Desarrollar metodología de reducción de tiempo de prueba utilizando APU12s en el ETS364\\
& & • Crear un método autónomo de caracterizando módulos de evaluación en el ensayador Eagle\\
& & • Ayudar integrar prueba de fuga de pins experimental para reguladores de baja caída\\
& & \\
\end{tabular}

%%%%%%%%%%%%%
% Education %
%%%%%%%%%%%%%
\begin{tabular}{ p{1.5cm} p{1cm} p{16cm} }
\textbf{Educación:} & & \textbf{Universidad McGill, Montréal, Quebec} \\
& & Máster en Ingeniería (M.Eng.), Ingenieria Eléctrica, Mayo 2017\\
& & Disertación: \textit{Análisis de la Capacidad de Sistemas Más-Rápido-que-Nyquist y MIMO}\\
& & \\
& & \textbf{Insituto de Tecnología Rose-Hulman, Terre Haute, Indiana} \\
& & Bachillar Universitario en Ciencias (B.Sc.), Ingenieria Eléctrica, Mayo 2013\\
& & Proyecto Final: \textit{Sistema de Visión de Vista Envolvente}\\
& & \\
\end{tabular}

%%%%%%%%%%%%%%%%
% Publications %
%%%%%%%%%%%%%%%%
\begin{tabular}{ p{1.5cm} p{1cm} p{16cm} }
\textbf{Publicaciones:} & & \textbf{On the Capacity of Faster-than-Nyquist MIMO Transmission with CSI at the Receiver}\\
& & 2015 IEEE Globecom Workshops (GC Wkshps)\\
& & Autores: Michael Yuhas, Yi Feng, Jan Bajcsy\\
& & \\
\end{tabular}

\end{document}
