\documentclass{minimal}
\usepackage{booktabs}
\usepackage{array}
\usepackage{lmodern}
\usepackage[letterpaper, margin=0.5in]{geometry}
\usepackage[utf8]{inputenc}
\renewcommand{\familydefault}{\sfdefault}
\begin{document}

%%%%%%%%%%%%%%%%
% Contact Info %
%%%%%%%%%%%%%%%%
\newcommand{\name}{First Last}
\newcommand{\address}{123 Evergreen Terrace\\Springfield, ND, 12345}
\newcommand{\phone}{1-234-567-8901}
\newcommand{\email}{somebody@someplace.com}
\newcommand{\github}{https://github.com/abcdef}

\begin{center}
\fontsize{14}{12.8}\selectfont
\name\\
\fontsize{10}{12}\selectfont
\address\\
\phone\\
\email\\
\github\\
\end{center}
\vspace{0.5cm}

%%%%%%%%%%
% Skills %
%%%%%%%%%%
\begin{tabular}{ p{1.5cm} p{1cm} p{16cm} }
\textbf{Compétences:} & & \textbf{Langages de Programmation:} Python, C, C++, Matlab, Java, LabVIEW\\
& & \textbf{Microcontrôleurs:} MSP430, PIC16, SAMv71, Cyclone II (Nios II), OMAP, Snapdragon\\
& & \textbf{Systèmes d'Exploitation Embarqués:} FreeRTOS, QNX, Yocto Linux, Angstrom Linux, Android\\
& & \textbf{Bibliothèques et Frameworks:} OpenCV, PIL, D3, Matplotlib, Apex Chart, Scikit-learn, Numpy\\
& & \textbf{Autres Compétences:} Visualisation de Données, Agile, Planification de Projet, Direction Technique\\
& & \\
\end{tabular}

%%%%%%%%%%%%%%
% Experience %
%%%%%%%%%%%%%%
\begin{tabular}{ p{1.5cm} p{1cm} p{16cm} >{\raggedleft\arraybackslash}p{3cm} }
\textbf{Expérience:} & & \textbf{Mojo Vision, Saratoga, Californie} & \\
\end{tabular}

\begin{tabular}{ p{1.5cm} p{1cm} p{16cm} }
& & \textit{Ingénieur Logiciel Senior d'État-Major, Logiciel, 2020}\\
& & • Établir processus de validation de logiciel de zéro; assurer conformité à ISO 13485\\
& & • 'Dockeriser' constructions d'intergiciel pour établir logiciel répétable et testé automatiquement\\
& & • Créer tests d'acceptation SIL et HIL dans le cadre du tube d'intégration continue\\
& & • Écrire bibliothèque en Python pour contrôler l'équipement HIL via UART et SSH\\
& & \\
\end{tabular}

\begin{tabular}{ p{1.5cm} p{1cm} p{16cm} >{\raggedleft\arraybackslash}p{3cm} }
& & \textbf{Nio, San José, Californie} & \\
\end{tabular}

\begin{tabular}{ p{1.5cm} p{1cm} p{16cm} }
& & \textit{Ingénieur Validation d'État-Major, Cockpit Numérique, 2018-2020}\\
& & • Chef de test pour le cockpit numérique de la prochaine génération; assurer conformité à ISO 26262\\
& & • Gérer équipe de sept ingénieurs pour quelques mois pendant que la gérante était malade\\
& & • Travailler étroitement avec vendeurs et équipe matérielle pour qualifier logiciel et matériel COTS\\
& & • Automatiser caractérisation des vitesses de lecture-écriture d'UFS à travers OSs et matériel\\
& & • Développer plugiciel de tableau de bord pour TestRail utilisant D3 pour visualiser couverture de test\\
& & • Visualiser résultats de tests de stress et longévité utilizant Matplotlib; analyser les données\\
& & • Entraîner nouveaux employés et équipes chinoises comment utiliser logiciel et matériel de test interne\\
& & • Créer bibliothèque en Python et matériel de mesure pour détecter et comparer signaux audio\\
& & • Créer bibliothèque en Python pour contrôler alimentations électriques du laboratoire à distance\\ 
& & • Guider équipe d'ingénieurs de automatisation à créer framework de test de mots-clés en Python\\
& & • Rechercher et planifier stratégie pour introduire model-based testing pour HMI et MCU\\
& & \\
\end{tabular}

\begin{tabular}{ p{1.5cm} p{1cm} p{16cm} }
& & \textit{Ingénieur Validation Senior, Cockpit Numérique, 2017-2018}\\
& & • Chef de test pour logiciel de fabrication; assurer qualité de logiciel de diagnostic\\
& & • Écrire tests de mémoires pour DRAM et eMMC pour identifier erreurs de retournement de bits\\
& & • Aider équipe de fabrication de Chine pour déboguer problèmes de test à distance utilisant LabVIEW\\
& & • Automatiser tests de longévité du système que dévoilaient bugs de démarrage critique pour la sécurité\\
& & • Travailler avec développeurs de logiciels et vendeurs pour déboguer problèmes de démarrage et de I2C\\
& & • Instaurer 50 bancs de test pour exécuter en continu tests de longévité et vérifier correctifs\\
& & • Utiliser OpenCV et Python pour détecter flux vidéo congelés et autres modes de défaillance de vidéo\\
& & • Créer bibliothèque en Python pour envoyer et recevoir messages LIN en utilisant l'outil PCAN-LIN\\
& & • Créer bibliothèque en Python pour communiquer avec trois OSs en dispositif avec un port de UART\\
& & • Mener stagiaire en projet de traitement d'image pour classifier images des bancs de test utilisant OpenCV\\
& & • Représenter bureau américain en Chine pour SUD, Salon de l'auto de Pékin, et événements en Shanghai\\
& & \\
\end{tabular}

\begin{tabular}{ p{1.5cm} p{1cm} p{10cm} >{\raggedleft\arraybackslash}p{3cm} }
& & \textbf{Apple, Cupertino, Californie} & \\
\end{tabular}

\begin{tabular}{ p{1.5cm} p{1cm} p{16cm} }
& & \textit{Ingénieur Logiciel d'Assurance Qualité, Équipe d'Applications, 2015-2017}\\
& & • Chef de test pour 'Actualités sur le Web'; tester l'extrémité avant et l'extrémité arrière\\ 
& & • Tester fonctionnalités d'ANF comme tableaux, parallaxe, et widgets des médias sociaux\\
& & • Automatiser tests de conversion des articles ANF pour trouver régressions entre mises à jour d'ANF\\
& & • Écrire outil en Python pour générer articles pour tester efficacement style et fonte de caractères\\
& & • Écrire outil en Python pour générer images pour couvrir tous les tests de traitement d'image\\
& & \\
\end{tabular}

% Extra Space to Bump the following to the next page
\pagebreak

\begin{tabular}{ p{1.5cm} p{1cm} p{10cm} >{\raggedleft\arraybackslash}p{3cm} }
& & \textbf{Université McGill, Montréal, Québec} & \\
\end{tabular}

\begin{tabular}{ p{1.5cm} p{1cm} p{16cm} }
& & \textit{Assistant Pédagogique, Département de Génie Électrique et Informatique, 2013-2015}\\
& & • Développer laboratoires de traitement du signal pour la FPGA Altera DE2 et NI myRIO\\
& & • Enseigner étudiants comment utiliser matériel de FPGA et concepts de conception numérique\\ 
& & • Implémenter filtres numériques, encodeurs convolutifs, et décodeurs Viterbi pour laboratoires\\
& & • Enseigner étudiants comment utiliser LabVIEW; l'utiliser pour implémenter filtres numériques et FSMs\\
& & • Faire un cours pour étudiants sur conception de système embarqué, réseaux de Petri, et réseaux de Kahn\\
& & \\
\end{tabular}

\begin{tabular}{ p{1.5cm} p{1cm} p{10cm} >{\raggedleft\arraybackslash}p{3cm} }
& & \textbf{Texas Instruments, Dallas, Texas} & \\
\end{tabular}

\begin{tabular}{ p{1.5cm} p{1cm} p{16cm} }
& & \textit{Ingénieur Produit Stagiaire, Operations d'Ingénierie Analogique, 2011-2013}\\
& & • Développer méthodologie de réduction de temps de test utilisant les APU12s pour le testeur ETS364\\
& & • Créer méthode autonome de caractérisant modules d'évaluation sur le testeur Eagle\\
& & • Aider à intégrer test de courant de fuite expérimental pour régulateurs linéaires\\
& & \\
\end{tabular}

%%%%%%%%%%%%%
% Education %
%%%%%%%%%%%%%
\begin{tabular}{ p{1.5cm} p{1cm} p{16cm} }
\textbf{Éducation:} & & \textbf{Université McGill, Montréal, Québec} \\
& & Master en Ingénierie (M.Eng.), Ingénierie Électrique, Mai 2017\\
& & Thèse: \textit{Analyse de la Capacité des Systèmes Plus-Rapide-que-Nyquist et MIMO}\\
& & \\
& & \textbf{Institut de Technologie Rose-Hulman, Terre Haute, Indiana} \\
& & Baccalauréat Universitaire en Sciences (B.Sc.), Ingénierie Électrique, Mai 2013\\
& & Projet Capstone: \textit{Système de Vision de Vue Entourée}\\
& & \\
\end{tabular}

%%%%%%%%%%%%%%%%
% Publications %
%%%%%%%%%%%%%%%%
\begin{tabular}{ p{1.5cm} p{1cm} p{16cm} }
\textbf{Publications:} & & \textbf{On the Capacity of Faster-than-Nyquist MIMO Transmission with CSI at the Receiver}\\
& & 2015 IEEE Globecom Workshops (GC Wkshps)\\
& & Auteurs: Michael Yuhas, Yi Feng, Jan Bajcsy\\
& & \\
\end{tabular}

\end{document}
